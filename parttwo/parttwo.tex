\documentclass{article}
\usepackage{color}
\usepackage{ulem}

\begin{document}
\title{Part Two: Sentence Structure}
\maketitle

In Part One we dealt with various types of unnecessary words. All the chapters in that part related to one central concern: the need to make an English sentence concise.

In Part Two we shall deal with \uwave{various aspects} of sentence structure. All the chapters in this part will relate to another central concern: the need to make an English sentence clear and logical.

The faults we shall consider in Part Two are, in general, harder to identify and harder to remedy than \uwave{those we studied} in Part One. But by the same token, they are more interesting. And once we have named and analyzed these mistakes, you will be able to recognize and correct them, just as you have learned to recognize and correct unnecessary words.

\section{The Noun Plague}

Plain English is a language based on verbs. It is simple, concise, vigorous and, above all, clear. Chinglish is a language based on nouns --- vague, general, abstract nouns. It is \uwave{complicated, long winded, ponderous, and obscure}.

As we saw in Chapter I, Chinglish contains many nouns that are unnecessary. We examined three types that contribute nothing to the meaning of a sentence and can simply be eliminated:

\begin{itemize}
  \item redundant nouns (``to accelerate the pace of economic reform'' = ``to accelerate economic reform''; `` there have been good harvests in agriculture'' $=$ ``there have been good harvests '')
  \item empty nouns (``following the realization of mechanization '' = `` following the mechanization''; ``we must pay attention to promotion'' $=$ ``we must promotee'')
  \item category nouns (``opposing the practice of extravagance '' $=$ ``opposing extravagance ''; ``to archive the objective of clarity'' = `` to be clear'')
\end{itemize}

At the same time, we looked at two groups of nouns that do carry necessary meaning but that drag unnecessary words along them. To express their meaning more concisely, we changed them to verbs:

\begin{itemize}
  \item nouns in the construction unnec. verb + noun (``to make an improvement in '' = ``to improve'')
  \item ``third word'' nouns in the construction unnec. verb + unnec. noun + third word (``to reach the goal of modernization'' = ``to modernize'')
\end{itemize}

There are also many constructions involving nouns like ``improvement'' and ``modernization'' that we did not discuss in Chapter I because they do not include unnecessary words. Yet those too should be edited out wherever possible, simply because the nouns are abstract.

In this chapter we shall consider abstract nouns as a class and see both why and how to avoid them.


\subsection{Perils of using abstract nouns}

Authorities on English consistently condemn the use of abstract language. The consensus is perhaps best summed up the American scholar Jacques Barzun, a master of the crafts of writing and translation. In his guide for writes, significantly entitled \emph{Simple and Direct}, he makes this recommendation:

Prefer the concrete word to the abstract. Follow that advice and you will see your prose gain in lucidity and force. Unnecessary abstraction is one of the worst faults of modern writing --- the string of nouns held together by prepositions and relying on the passive voice to convey the enfeebled sense

In the same way, Ernest Gowers, addressing British civil servants, singles out the preference for the abstract word as ``the greatest vice of present-day writing.'' He warns in particular the ``an excessive reliance on the noun at the expense of the verb will ... insensibly induce a habit of abstraction, generalisation and vagueness.''

The authors of books on writing often use metaphors comparing the use of abstract noun to infection. Fowler calls it ``disease'' of ``abstractitis.'' William Zinsser speaks of ``dead'' sentences and the blight of ``creeping nounism''. And Wilson Follet, whose \emph{Modern American Usage} is as much a classic as Fowler's \emph{Dictionary of Modern English Usage}, urges writers
to ``avoid abstract nouns like the plague.''

\subsubsection{Sentences based on abstract nouns}
To see how abstract nouns undermine straightforward communication we have only to look at the following example, in which the same idea is expressed in two different ways:

\begin{itemize}
  \item A: The prolongation of the existence of this temple is due to the solidity of its construction.
  \item B: This temple \emph{has endured} because it was solidly built.
\end{itemize}

The first version contain four abstract nouns, while the successive has none. Not only do the nouns make the statement nearly twice as long, but they also make it pretentious, wooden, and hard to understand.

Chinglish abounds in sentences that rely chiefly on nouns to express their meaning. Here are three examples taken from draft translations:

\begin{itemize}
  \item A curtailment of both city and town population as well as streamlining of administrative personnel for an enhancement of work efficiency constitutes a important aspect in our tasks of adjustment.
  \item With stability of currency and prices achieved, there had to be proved communications throughout the country, in accordance with the new conditions and demands, so that they could serve the restoration of production.
  \item The basic requirements of the struggle against Right-leaning conservatism are the further development and consolidation of the people's dictatorship in our country, the early accomplishment of socialist transformation, the overfulfillment of the state plan for industrial development, and the rapid progress of the technical transformation of the national economy.
\end{itemize}

The preference for abstract language is not of course, unique to Chinglish. If the advice-givers make such a point of condemning it, that is precisely because it pervades so much of the writing of native speaker of English.

Abstract nouns are the common coin of academic institutions, government bureaucracies, the military establishment, large corporations, and so on (groups for which obscurity if often advantage). Abstract nouns are also favored by man ordinary citizens, when they wish to sound authoritative or scientific.

Following are three more examples, this time from the English produced by native speakers, of what has been termed "the noun style." Like the Chinglish sentences cited above, they illustrate what Gowers calls "the habit of using abstract words to say in a complicated way something that might be said simply and directly."



\end{document}
